\documentclass[a4paper, 11pt]{article}

%Comandos para configurar el idioma
\usepackage[spanish,activeacute]{babel}
\usepackage[utf8]{inputenc}
\usepackage[T1]{fontenc} %Necesario para el uso de las comillas latinas.

%Importante que esta sea la última órden del preámbulo
\usepackage{hyperref}
\hypersetup{
  pdftitle={Cuestionario de teoría - 3},
  pdfauthor={Alejandro García Montoro},
  unicode,
  plainpages=false,  colorlinks,
  citecolor=black,
  filecolor=black,
  linkcolor=black,
  urlcolor=black,
}
\newcommand\fnurl[2]{%
  \href{#2}{#1}\footnote{\url{#2}}%
}

%Paquetes matemáticos
\usepackage{amsmath,amsfonts,amsthm}
%\usepackage[all]{xy} %Para diagramas
\usepackage{enumerate} %Personalización de enumeraciones
\usepackage{tikz} %Dibujos

%Tipografía escalable
\usepackage{lmodern}
%Legibilidad
\usepackage{microtype}

%Código
\usepackage{listings}
\usepackage{color}

\definecolor{dkgreen}{rgb}{0,0.6,0}
\definecolor{gray}{rgb}{0.5,0.5,0.5}
\definecolor{mauve}{rgb}{0.58,0,0.82}

\lstset{frame=tb,
  language=Python,
  aboveskip=3mm,
  belowskip=3mm,
  showstringspaces=false,
  columns=flexible,
  basicstyle={\small\ttfamily},
  numbers=left,
  numberstyle=\tiny\color{gray},
  keywordstyle=\color{blue},
  commentstyle=\color{dkgreen},
  stringstyle=\color{mauve},
  breaklines=true,
  breakatwhitespace=true,
  tabsize=3
}

\title{Cuestionario de teoría \\ 3}
\author{Alejandro García Montoro\\
    \href{mailto:agarciamontoro@correo.ugr.es}{agarciamontoro@correo.ugr.es}}
\date{\today}

\theoremstyle{definition}
\newtheorem{ejercicio}{Ejercicio}
\newtheorem*{solucion}{Solución}

\theoremstyle{theorem}
\newtheorem{cuestion}{Cuestión}
\newtheorem{bonus}{Bonus}
\newtheorem{teorema}{Teorema}

\begin{document}

    \maketitle

    \section{Cuestiones}

    \begin{ejercicio}
        En clase se ha mostrado una técnica para estimar el vector de traslación del movimiento de un par estéreo y sólo ha podido estimarse su dirección. Argumentar de forma lógica a favor o en contra del hecho de que dicha restricción sea debida a la técnica usada o sea un problema inherente a la reconstrucción.
    \end{ejercicio}

    \begin{solucion}
        El problema es inherente a la reconstrucción, y viene de la matriz esencial $E$. Esta sólo puede ser conocida salvo un factor de escala, por lo que al intentar
    \end{solucion}

    \begin{ejercicio}
        Verificar matemáticamente que se deben de cumplir las ecuaciones $Fe = 0$, and $F^T e' = 0$.
    \end{ejercicio}

    \begin{solucion}
        Partimos de que todas las líneas epipolares $l'$ contienen al epipolo $e'$; es decir:
        \begin{equation}
            e'^Tl' = 0
            \label{eq:epipole}
        \end{equation}

        Además, de la definición de línea epipolar, tenemos que
        \begin{equation}
            l' = Fp
            \label{eq:defLine}
        \end{equation}
        con $p$ un punto imagen izquierda.

        Sustituyendo \ref{eq:defLine} en \ref{eq:epipole}, tenemos que
        \[
            e'^TFp = 0
        \]

        Como lo anterior se cumple para todo $p$, concluimos que
        \[
            e'^TF = 0
        \]
        y, por tanto, llegamos a la igualdad buscada:
        \[
            F^T e' = 0
        \]

        De la misma forma, con $q$ un punto imagen derecha, sabemos que
        \[
            e^T F^T q = 0
        \]

        Razonando como antes, como $q$ es arbitrario, podemos concluir que
        \[
            e^T F^T = 0
        \]
        y obtenemos la otra igualdad:

        \[
            F e = 0
        \]


    \end{solucion}

    \begin{ejercicio}
        Verificar matemáticamente que cuando una cámara se desplaza las coordenadas retina de puntos correspondientes sobre la retina deben de verificar la ecuación $x'= x + \frac{Kt}{Z}$.
    \end{ejercicio}

    \begin{solucion}

    \end{solucion}

    \begin{ejercicio}
        Dar una interpretación geométrica a las columnas y filas de la matriz $P$ de una cámara.
    \end{ejercicio}

    \begin{solucion}
        En \emph{Multiple View Geometry in Computer Vision}, de Hartley y Zisserman, se da una descripción muy precisa de cada fila y columna de una matriz cámara $P$. Denotando $P^i$ a la i-ésima fila de la matriz y $p_j$ a la j-ésima columna, tenemos lo siguiente:
        \begin{itemize}
            \item Las tres primeras columnas, $p_1$, $p_2$ y $p_3$ son puntos vanishing en la imagen que corresponden, respectivamente, a los ejes $X$, $Y$ y $Z$. La \item La última columna, $p_4$, es la imagen del origen de coordenadas.
            \item $P^1$ y $P^2$, primera y segunda filas de $P$, son planos en el espacio que contienen al centro de la cámara, y corresponden a los puntos cuya proyección se encuentra, respectivamente, en las líneas $x=0$ e $y=0$
            \item La última fila de $P$, $P^3$, es el llamado plano principal de la cámara; es decir, el plano que pasa por el centro de la cámara y es paralelo al plano imagen.
        \end{itemize}
    \end{solucion}

    \begin{ejercicio}
        Suponga una matriz $A(3\times3)$ de números reales. Suponga $rango(A)=3$. ¿Cuál es la matriz esencial más cercana a $A$ en norma de Frobenius? Argumentar matemáticamente la derivación.
    \end{ejercicio}

    \begin{solucion}

    \end{solucion}

    \begin{ejercicio}
        Discutir cuáles son las ventajas y desventajas de usar un algoritmo de reconstrucción euclídea que calcule la profundidad de varios puntos a la vez en lugar de uno a uno.
    \end{ejercicio}

    \begin{solucion}

    \end{solucion}

    \begin{ejercicio}
        Deducir la expresión para la matriz esencial $E = [t]x R = R[RTt]x$. Justificar cada uno de los pasos.
    \end{ejercicio}

    \begin{solucion}

    \end{solucion}

    \begin{ejercicio}
        Dada una pareja de cámaras cualesquiera, ¿existen puntos del espacio que no tengan un plano epipolar asociado? Justificar la respuesta.
    \end{ejercicio}

    \begin{solucion}

    \end{solucion}

    \begin{ejercicio}
        Si nos dan las coordenadas de proyección de un punto escena en la cámara\_1 y nos dicen cuál es el movimiento relativo de la cámara\_2 respecto de la cámara\_1, ¿es posible reconstruir la profundidad el punto si las cámaras están calibradas? Justificar la contestación
    \end{ejercicio}

    \begin{solucion}

    \end{solucion}

    \begin{ejercicio}
        Suponga que obtiene una foto de una escena y la cámara gira para obtener otra foto. ¿Cuál es la ecuación que liga las coordenadas de las proyecciones en ambas imágenes, de un mismo punto escena, en términos de los parámetros de las cámaras? Justificar matemáticamente.
    \end{ejercicio}

    \begin{solucion}

    \end{solucion}

    \begin{ejercicio}
        Suponga una cámara afín. Discutir cuáles son los efectos de la proyección ortogonal sobre los parámetros intrínsecos y extrínsecos de la cámara.
    \end{ejercicio}

    \begin{solucion}

    \end{solucion}

    \begin{ejercicio}
        Dadas dos cámaras calibradas, $P=K[I|0]$ y $P'=K'[R|t]$, calcular la expresión de la matriz fundamental en términos de los parámetros intrínsecos y extrínsecos de las cámaras. Todos los pasos deben ser justificados.
    \end{ejercicio}

    \begin{solucion}

    \end{solucion}
\end{document}
