\documentclass[a4paper, 11pt]{article}

%Comandos para configurar el idioma
\usepackage[spanish,activeacute]{babel}
\usepackage[utf8]{inputenc}
\usepackage[T1]{fontenc} %Necesario para el uso de las comillas latinas.

%Importante que esta sea la última órden del preámbulo
\usepackage{hyperref}
\hypersetup{
  pdftitle={Cuestionario de teoría - 3},
  pdfauthor={Alejandro García Montoro},
  unicode,
  plainpages=false,  colorlinks,
  citecolor=black,
  filecolor=black,
  linkcolor=black,
  urlcolor=black,
}
\newcommand\fnurl[2]{%
  \href{#2}{#1}\footnote{\url{#2}}%
}

%Paquetes matemáticos
\usepackage{amsmath,amsfonts,amsthm}
%\usepackage[all]{xy} %Para diagramas
\usepackage{enumerate} %Personalización de enumeraciones
\usepackage{tikz} %Dibujos

%Tipografía escalable
\usepackage{lmodern}
%Legibilidad
\usepackage{microtype}

%Código
\usepackage{listings}
\usepackage{color}

\definecolor{dkgreen}{rgb}{0,0.6,0}
\definecolor{gray}{rgb}{0.5,0.5,0.5}
\definecolor{mauve}{rgb}{0.58,0,0.82}

\lstset{frame=tb,
  language=Python,
  aboveskip=3mm,
  belowskip=3mm,
  showstringspaces=false,
  columns=flexible,
  basicstyle={\small\ttfamily},
  numbers=left,
  numberstyle=\tiny\color{gray},
  keywordstyle=\color{blue},
  commentstyle=\color{dkgreen},
  stringstyle=\color{mauve},
  breaklines=true,
  breakatwhitespace=true,
  tabsize=3
}

\title{Cuestionario de teoría \\ 3}
\author{Alejandro García Montoro\\
    \href{mailto:agarciamontoro@correo.ugr.es}{agarciamontoro@correo.ugr.es}}
\date{\today}

\theoremstyle{definition}
\newtheorem{ejercicio}{Ejercicio}
\newtheorem*{solucion}{Solución}

\theoremstyle{theorem}
\newtheorem{cuestion}{Cuestión}
\newtheorem{bonus}{Bonus}
\newtheorem{teorema}{Teorema}

\begin{document}

    \maketitle

    \section{Cuestiones}

    \begin{ejercicio}
        En clase se ha mostrado una técnica para estimar el vector de traslación del movimiento de un par estéreo y sólo ha podido estimarse su dirección. Argumentar de forma lógica a favor o en contra del hecho de que dicha restricción sea debida a la técnica usada o sea un problema inherente a la reconstrucción.
    \end{ejercicio}

    \begin{solucion}
        El problema es inherente a la reconstrucción. Supongamos la siguiente situación: un punto escena $P$ se encuentra a distancia $d_{izqda}$ de la cámara de la izquierda y a distancia $d_{dcha}$ de la de la derecha. Ambas cámaras se encuentran a distancia $d$ la una de la otra.

        Supongamos que tenemos las proyecciones del punto $P$ por las dos cámaras pero no sabemos ninguna de las distancias. Es evidente que en una situación con distancias $d_{izqda}$, $d_{dcha}$ y $d$ se producirán exactamente las mismas proyecciones ---podemos argumentar esto por simple semejanza de triángulos--- que en una situación con distancias $kd_{izqda}$, $kd_{dcha}$ y $kd$, con $k\in\mathbb{R}^+$.

        Es imposible por tanto conocer el módulo del cálculo del vector de traslación, cuya dirección sin embargo sí conocemos, ya que esto correspondería a conocer, con la notación que hemos usado, la distancia $d$. Y esto es imposible, pues de antemano no conocemos ni $d_{izqda}$ ni $d_{dcha}$.
    \end{solucion}

    \begin{ejercicio}
        Verificar matemáticamente que se deben de cumplir las ecuaciones $Fe = 0$, and $F^T e' = 0$.
    \end{ejercicio}

    \begin{solucion}
        Partimos de que todas las líneas epipolares $l'$ contienen al epipolo $e'$; es decir:
        \begin{equation}
            e'^Tl' = 0
            \label{eq:epipole}
        \end{equation}

        Además, de la definición de línea epipolar, tenemos que
        \begin{equation}
            l' = Fp
            \label{eq:defLine}
        \end{equation}
        con $p$ un punto imagen izquierda.

        Sustituyendo \ref{eq:defLine} en \ref{eq:epipole}, tenemos que
        \[
            e'^TFp = 0
        \]

        Como lo anterior se cumple para todo $p$, concluimos que
        \[
            e'^TF = 0
        \]
        y, por tanto, llegamos a la igualdad buscada:
        \[
            F^T e' = 0
        \]

        De la misma forma, con $q$ un punto imagen derecha, sabemos que
        \[
            e^T F^T q = 0
        \]

        Razonando como antes, como $q$ es arbitrario, podemos concluir que
        \[
            e^T F^T = 0
        \]
        y obtenemos la otra igualdad:

        \[
            F e = 0
        \]


    \end{solucion}

    \begin{ejercicio}
        Verificar matemáticamente que cuando una cámara se desplaza las coordenadas retina de puntos correspondientes sobre la retina deben de verificar la ecuación $x'= x + \frac{Kt}{Z}$.
    \end{ejercicio}

    \begin{solucion}

    \end{solucion}

    \begin{ejercicio}
        Dar una interpretación geométrica a las columnas y filas de la matriz $P$ de una cámara.
    \end{ejercicio}

    \begin{solucion}
        En \emph{Multiple View Geometry in Computer Vision}, de Hartley y Zisserman, se da una descripción muy precisa de cada fila y columna de una matriz cámara $P$. Denotando $P^i$ a la i-ésima fila de la matriz y $p_j$ a la j-ésima columna, tenemos lo siguiente:
        \begin{itemize}
            \item Las tres primeras columnas, $p_1$, $p_2$ y $p_3$ son puntos vanishing en la imagen que corresponden, respectivamente, a los ejes $X$, $Y$ y $Z$. La \item La última columna, $p_4$, es la imagen del origen de coordenadas.
            \item $P^1$ y $P^2$, primera y segunda filas de $P$, son planos en el espacio que contienen al centro de la cámara, y corresponden a los puntos cuya proyección se encuentra, respectivamente, en las líneas $x=0$ e $y=0$
            \item La última fila de $P$, $P^3$, es el llamado plano principal de la cámara; es decir, el plano que pasa por el centro de la cámara y es paralelo al plano imagen.
        \end{itemize}
    \end{solucion}

    \begin{ejercicio}
        Suponga una matriz $A(3\times3)$ de números reales. Suponga $rango(A)=3$. ¿Cuál es la matriz esencial más cercana a $A$ en norma de Frobenius? Argumentar matemáticamente la derivación.
    \end{ejercicio}

    \begin{solucion}

    \end{solucion}

    \begin{ejercicio}
        Discutir cuáles son las ventajas y desventajas de usar un algoritmo de reconstrucción euclídea que calcule la profundidad de varios puntos a la vez en lugar de uno a uno.
    \end{ejercicio}

    \begin{solucion}
        Cabría pensar en un primer momento que con un algoritmo de ese tipo mejoraríamos la precisión del cálculo de la profundidad de cada punto. Sin embargo, esto no tiene ningún sentido: las profundidades de los puntos son independientes las unas de las otras, y la precisión de su cálculo no se verá lógicamente afectada por el aumento de puntos.

        Sin embargo, el calcular profundidades lo podemos usar para algo en principio diferente: así, cuantos más puntos tengamos en cuenta a la vez, mejor será la aproximación de los parámetros extrínsecos de la cámara. La justificación de esto es clara: cuantos más puntos tengamos, mejor será la estimación de la matriz y por tanto de los parámetros.

        Las desventajas de un algoritmo así son claras: cuantos más puntos haya, más costosos serán todos los cálculos y, aunque ganemos en precisión, perderemos en eficiencia.
    \end{solucion}

    \begin{ejercicio}
        Deducir la expresión para la matriz esencial $E = [t]_x R = R[R^Tt]_x$. Justificar cada uno de los pasos.
    \end{ejercicio}

    \begin{solucion}
        Sabemos que
        \[
        E = K^T F K
        \]

        Por otro lado, en el ejercicio 12 hemos deducido que $F = [K't]_xK'RK^{-1}$. Por tanto, sustituyendo en la definición de $E$ tenemos:
        \begin{align*}
            E &= K^T [K't]_xK'RK^{-1} K = \\
            &= K^T [K't]_xK'R =\\
            &= [t]_x R
        \end{align*}
    \end{solucion}

    \begin{ejercicio}
        Dada una pareja de cámaras cualesquiera, ¿existen puntos del espacio que no tengan un plano epipolar asociado? Justificar la respuesta.
    \end{ejercicio}

    \begin{solucion}

    \end{solucion}

    \begin{ejercicio}
        Si nos dan las coordenadas de proyección de un punto escena en la cámara 1 y nos dicen cuál es el movimiento relativo de la cámara 2 respecto de la cámara 1, ¿es posible reconstruir la profundidad el punto si las cámaras están calibradas? Justificar la contestación
    \end{ejercicio}

    \begin{solucion}
        No, para reconstruir el punto escena $p$ necesitaríamos su proyección en la segunda cámara.

        Aunque podemos obtener la segunda cámara a partir de la primera, pues sabemos su movimiento relativo, para realizar el cálculo del punto escena necesitamos los dos rayos que nos definan la intersección donde está $p$.

        El primer rayo, correspondiente a la primera imagen, lo tenemos; pero no así el segundo, pues aunque conocemos la segunda cámara no conocemos la proyección de $p$ por ella.
    \end{solucion}

    \begin{ejercicio}
        Suponga que obtiene una foto de una escena y la cámara gira para obtener otra foto. ¿Cuál es la ecuación que liga las coordenadas de las proyecciones en ambas imágenes, de un mismo punto escena, en términos de los parámetros de las cámaras? Justificar matemáticamente.
    \end{ejercicio}

    \begin{solucion}

    \end{solucion}

    \begin{ejercicio}
        Suponga una cámara afín. Discutir cuáles son los efectos de la proyección ortogonal sobre los parámetros intrínsecos y extrínsecos de la cámara.
    \end{ejercicio}

    \begin{solucion}

    \end{solucion}

    \begin{ejercicio}
        Dadas dos cámaras calibradas, $P=K[I|0]$ y $P'=K'[R|t]$, calcular la expresión de la matriz fundamental en términos de los parámetros intrínsecos y extrínsecos de las cámaras. Todos los pasos deben ser justificados.
    \end{ejercicio}

    \begin{solucion}
        De nuevo, Hartley y Zisserman desarrollan varios cálculos en los que obtienen la matriz fundamental a partir de diversas formas. Entre ellas, se encuentra la siguiente deducción:

        Llamando $P^+$ a la pseudo-inversa de $P$; es decir, aquella matriz tal que $P^+ P = I$, su expresión es
        \[
        P^+ = \left[\begin{array}{c}
            K^{-1} \\
            0^T
        \end{array}\right]
        \]

        EL centro de la cámara $P$, definido por $PC = 0$ es
        \[
        C = \left(\begin{array}{c}
            0 \\
            1
        \end{array}\right)
        \]

        Si suponemos la familia de soluciones para $X$ de la ecuación $PX = x$, con $x$ punto de la imagen, podemos coger dos puntos que se encuentran en el rayo solución; a saber: $P^+x$ y el centro de la cámara $P$: el punto $C$.

        Las proyecciones de estos dos puntos por la cámara $P'$ nos da dos puntos: $P'P^+x$ y $P'C$, que definen la línea epipolar como sigue:
        \[
            l' = (P'C) \times (P'P^+x)
        \]

        Podemos escribir la ecuación anterior como
        \[
            l' =[P'C]_x(P'P^+)x
        \]

        Esto nos da la definición de matriz fundamental, $F = [P'C]_x(P'P^+)$.

        Podemos ahora sustituir $P$ y $P'$ por las definiciones del enunciado y operar:
        \begin{align*}
            F &= [P'C]_x(P'P^+) = \\
            &= [(K'[R|t])C]_x(K'[R|t]\left[\begin{array}{c}
                K^-1 \\
                0^T
            \end{array}\right]) = \\
            &= [K't]_xK'RK^{-1}
        \end{align*}
        llegando así a una expresión de $F$ en función únicamente de $K$, parámetros intrínsecos de la cámara y $R$ y $t$, parámetros extrínsecos.
    \end{solucion}
\end{document}
