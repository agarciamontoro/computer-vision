\documentclass[a4paper, 11pt]{article}

%Comandos para configurar el idioma
\usepackage[spanish,activeacute]{babel}
\usepackage[utf8]{inputenc}
\usepackage[T1]{fontenc} %Necesario para el uso de las comillas latinas.

%Importante que esta sea la última órden del preámbulo
\usepackage{hyperref}
\hypersetup{
  pdftitle={Informe de prácticas - 1},
  pdfauthor={Alejandro García Montoro},
  unicode,
  plainpages=false,
  colorlinks,
  citecolor=black,
  filecolor=black,
  linkcolor=black,
  urlcolor=black,
}
\newcommand\fnurl[2]{%
  \href{#2}{#1}\footnote{\url{#2}}%
}

%Paquetes matemáticos
\usepackage{amsmath,amsfonts,amsthm}
\usepackage{enumerate} %Personalización de enumeraciones
\usepackage{tikz} %Dibujos

%Tipografía escalable
\usepackage{lmodern}
%Legibilidad
\usepackage{microtype}

%Código
\usepackage{listings}
\usepackage{color}

\definecolor{dkgreen}{rgb}{0,0.6,0}
\definecolor{gray}{rgb}{0.5,0.5,0.5}
\definecolor{mauve}{rgb}{0.58,0,0.82}

\lstset{frame=tb,
  language=C++,
  aboveskip=3mm,
  belowskip=3mm,
  showstringspaces=false,
  columns=flexible,
  basicstyle={\small\ttfamily},
  numbers=left,
  numberstyle=\tiny\color{gray},
  keywordstyle=\color{blue},
  commentstyle=\color{dkgreen},
  stringstyle=\color{mauve},
  breaklines=true,
  breakatwhitespace=true,
  tabsize=3
}

\title{Cuestionario de teoría \\ 1}
\author{Alejandro García Montoro\\
    \href{mailto:agarciamontoro@correo.ugr.es}{agarciamontoro@correo.ugr.es}}
\date{\today}

\theoremstyle{definition}
\newtheorem{ejercicio}{Ejercicio}
\newtheorem*{solucion}{Solución}

\theoremstyle{theorem}
\newtheorem{cuestion}{Cuestión}
\newtheorem{bonus}{Bonus}
\newtheorem{teorema}{Teorema}

\begin{document}

  \maketitle

  \section{Análisis de la implementación}

  \subsection{Función de convolución}

  La implementación de la función de convolución 2D se construye sobre funciones más simples; a saber:
  \begin{itemize}
      \item Cálculo de una máscara gaussiana 1D.
      \item Convolución 1D con una máscara general.
  \end{itemize}

  Sobre la convolución 2D se construye, además, la función lowPassFilter(), que implementa un filtro Gaussiano.

  \subsubsection{Cálculo de una máscara gaussiana 1D}
  La máscara gaussiana se construye con la función getGaussMask(), cuyo código se puede ver más abajo. Esta función devuelve un objeto Mat que representa un vector uni-dimensional generado a partir de la discretización de la función gaussiana en el intervalo $[-3\sigma, 3\sigma]$.

  El tamaño de la máscara tiene que ser un número natural impar. Así, primero se redondea el resultado de triplicar el $\sigma$, luego se duplica lo anterior y por último se le añade uno.

  Además, las máscaras de convolución tienen una restricción: la suma de sus valores tiene que ser igual a uno. Por tanto, primero se genera la máscara discretizando la función sin atender a esta restricción y luego se \emph{normaliza}; esto es, se divide cada valor de la máscara por la suma de todos ellos.

  \begin{lstlisting}
      /**
       * Builds a gaussian mask given the parameter sigma. The gaussian function is
       * sampled in the interval [-3*sigma, 3*sigma].
       */
      Mat Image::getGaussMask(double sigma){
          // Gaussian function needs to be sampled between -3*sigma and +3*sigma
          // and the mask has to have an odd dimension.
          int mask_size = 2*round(3*sigma) + 1;

          Mat gauss_mask = Mat(1,mask_size,CV_32FC1);

          // It is necessary to normalize the mask, so the sum of its elements
          // needs to be saved.
          float values_sum = 0;

          // Fills the mask with a sampled gaussian function and saves the sum of all elements
          for (int i = 0; i < mask_size; i++) {
              gauss_mask.at<float>(0,i) = gaussianFunction(i-mask_size/2, sigma);
              values_sum += gauss_mask.at<float>(0,i);
          }

          // Normalizes the gauss mask.
          gauss_mask = gauss_mask / values_sum;

          return gauss_mask;
      }
  \end{lstlisting}

  Los valores de la función gaussiana se consiguen con la siguiente implementación, que da el valor de la función en un punto $x$ con parámetro $\sigma$.
  \begin{lstlisting}
      /**
       * Samples the 1D gaussian function at point x with parameter sigma
       */
      double Image::gaussianFunction(double x, double sigma){
          return exp(-0.5*(x*x)/(sigma*sigma));
      }
  \end{lstlisting}

  \subsubsection{Convolución 1D con una máscara general}

  El siguiente paso es la implementación de una convolución uni-dimensional con una máscara general. El código es como sigue:

  \begin{lstlisting}
      /**
       * Returns the result of convolving the uni-dimensional signal_vec with the
       * mask, applying one of two types of borders: REFLECT or ZEROS.
       */
      Mat Image::convolution1D(const Mat& signal_vec, const Mat& mask, enum border_id border_type){
          assert(signal_vec.rows == 1 && mask.rows == 1 && mask.cols < signal_vec.cols);

          int num_channels = signal_vec.channels();

          // Initialization of source vector with additional borders.
          int border_size = mask.cols/2;  // Number of pixels added to each side
          Mat bordered;
          copyMakeBorder(signal_vec,bordered,0,0,border_size,border_size,border_type,0.0);

          // Splitting of the bordered vector for making a per-channel processing
          vector<Mat> bordered_channels(num_channels);
          split(bordered, bordered_channels);

          // Declaration of the result vector -with same size and type as the
          // original signal vector- and its splitted channels.
          Mat result = Mat(signal_vec.size(), signal_vec.type());
          vector<Mat> result_channels(num_channels);
          split(result, result_channels);

          // The mask and the source/result channels need to have the same type.
          // They are all converted to CV_32FC1 in order not to lose precision.
          Mat converted_mask;
          mask.convertTo(converted_mask,CV_32FC1);

          // Per-channel processing: we need the source channels, the masked channels;
          // i.e., the source channel focused in a ROI of the same size as the mask
          // and the result channels.
          Mat source_channel, masked_channel, result_channel;

          for (int i = 0; i < num_channels; i++) {
              // Channel type conversion
              bordered_channels[i].convertTo(source_channel, CV_32FC1);
              result_channels[i].convertTo(result_channel, CV_32FC1);

              // Actual processing
              for (int j = 0; j < result.cols; j++) {
                  // We focus on the zone centered at j with mask width
                  masked_channel = source_channel(Rect(j,0,mask.cols,1));

                  // Scalar product between the ROI'd source and the mask
                  result_channel.at<float>(0,j)  = masked_channel.dot(converted_mask);
              }

              // Backwards conversion: the result should have the same type as the input image
              result_channel.convertTo(result_channels[i],result_channels[i].type());
          }

          // Merging again the processed channels
          merge(result_channels, result);

          return result;
      }
  \end{lstlisting}

  El tener que tratar todos los canales de la imagen por separado y después hacer de nuevo la unión hace el código algo más difícil de leer, pero la idea es sencilla:
  \begin{enumerate}
      \item Se genera un vector señal con bordes (replicados o a ceros, según la decisión del usuario). El tamaño de estos bordes es igual a la mitad del tamaño de la máscara menos uno.
      \item Para cada píxel del vector señal con bordes
  \end{enumerate}

  \section{Análisis de resultados}
\end{document}
